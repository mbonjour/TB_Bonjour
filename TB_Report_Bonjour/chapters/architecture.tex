\chapter{Architecture / Design du protocole}
% TODO : s'inspirer de PRIVIpk ET EXPLIQUER LES BESOINS EN MATIèRE DE SéCURITé
% TODO Spécifications du protocole en texte via chapitre ou autre plus précise
\label{ch:arch}
Dans ce chapitre, je vais m'intéresser à expliquer le fonctionnement de la \textit{certificateless cryptography} et démontrer comment je l'ai utilisée afin de l'intégrer à un protocole de chiffrement de mail.
\section{Architecture globale}
Dans la Figure \ref{fig:globalProtocol}, je présente uniquement l'architecture globale pour bien représenter les différents acteurs présents dans le protocole et ainsi avoir une vue d'ensemble pour faciliter la compréhension.

\begin{figure}[h!]
	\centering
	\includegraphics[width=14cm]{images/SchemaGlobal.png}
	\caption{Schéma global du protocole}
	\label{fig:globalProtocol}
\end{figure}

%TODO compléter ce chapitre et préciser les actions des acteurs plus en détails
\section{Acteurs}
Les parties impliquées sont les suivantes comme vu à la Figure \ref{fig:globalProtocol}.
\begin{itemize}
	\item Alice : L'envoyeur du mail en direction de Bob. Alice doit discuter avec le KGC pour construire sa clé privée (afin de signer) et récupérer la clé publique de Bob.
	\item Bob : Le destinataire du message, communique uniquement avec le KGC en ayant reçu le message d'Alice afin de récupérer sa clé publique pour vérifier la signature et construire sa clé privée pour déchiffrer le message.
	\item KGC : Permet aux différents acteurs de pouvoir récupérer les clés publiques des clients, mais aussi de  recevoir les \textit{Partial Private Keys} qui permettent aux acteurs de construire leur clé privée. 
\end{itemize}
Ces différents acteurs sont les principaux présents dans un exemple de \textit{Certificateless Cryptography} dans le système de messagerie implémenté par la suite, mais on pourrait imaginer un serveur gérant uniquement les clés publiques.
\section{Fonctionnement Certificateless PKC}
Je vais ici découper les différents algorithmes présents dans le certificateless public key cryptography. En passant par le chiffrement et la signature.

Ces algorithmes seront accompagnés d'explications sur leur utilité. Les noms donnés aux algorithmes seront réutilisés ensuite pour les schémas afin de démontrer l'architecture du protocole mis en place. On peut voir des définitions spécifiques dans l'article sur lequel je me suis appuyé pour ce travail~\cite{conf/pkc/DentLP08}.
\subsection{Chiffrement}
La liste des différents algorithmes de \textit{Certificateless Cryptography} et leur description, les détails techniques de leurs implémentations sont disponibles à la Section \ref{sec:primitiveSearch}.
\begin{itemize}
	\item \textit{Setup.} (seulement une fois par le KGC).
	\item \textit{Partial-Private-Key-Extract.} Calcul d'une clé privée partielle lorsqu'un client le demande pour identité donnée.
	\item \textit{Set-Secret-Value.} Le client ne le fait qu'une fois pour tirer sa valeur secrète.
	\item \textit{Set-Private-Key.} Le client combine ses clés partielles et sa clé secrète pour obtenir une clé privée afin de déchiffrer les messages reçus, chiffrés avec une certaine identité.
	\item \textit{Set-Public-Key.} Le client ne le fait qu'une fois, il calcule sa clé publique en fonction de sa valeur secrète.
	\item \textit{Encrypt.} Chiffre un message avec la clé publique du destinataire et son identité.
	\item \textit{Decrypt.} Déchiffre un message utilisant sa clé privée et l'identité utilisée pendant le chiffrement.
\end{itemize}
\subsection{Signature}
Pour la signature, les algorithmes sont les mêmes avec une différence dans leur conception et évidemment le \textit{Encrypt} et \textit{Decrypt} sont remplacé par \textit{Sign} et \textit{Verify}.
%TODO compléter voir tableau

Dans la littérature certificateless, les schémas de signatures sont apparemment beaucoup plus cassés que ceux de chiffrement (voir tableau en Annexe \ref{ch:fichiers}). Il faut donc faire attention à vérifier régulièrement les différents schémas afin de vérifier que le schéma choisi ne soit pas mis à mal.
\section{Design du protocole}
Dans cette section le fonctionnement du protocole est décrit à l'aide de diagramme de séquences. Des notations sont utilisées afin de séparer correctement les paramètres utiles au chiffrement sont dénotés d'un E alors que les paramètres pour la signature sont dénotés d'un S. Pour les prochaines figures présentées on peut supposer que le KGC a été initialisé auparavant.

\subsection{Premier contact}
Description du premier contact effectué avec le KGC. 
\begin{figure}
[h!]
	\centering
	\begin{sequencediagram}
		\newthread{A}{Alice}{}
		\newinst[8]{B}{KGC}{} 
		\begin{call}{A}{Initialisation with alice@mail.ch}{B}{OK, $mpk_E, mpk_S$}
		\end{call}
	\postlevel
		\begin{callself}{A}{\shortstack{SetSec $x_E = Z_p^*$ \\ SetSecSig $x_S = Z_p^*$}}{}
		\end{callself}
	\postlevel
		\begin{callself}{A}{\shortstack{SetPub $PKE_{Alice} = (g^{x_E}, g_{1}^{x_E})$\\SetPubSig $PKS_{Alice} =x_SP$}}{}
		\end{callself}
	\postlevel
		\begin{call}{A}{$PKE_{Alice}, PKS_{Alice}$}{B}{}
		\end{call}
		
	\end{sequencediagram}
	\caption{Schéma de la première connexion}
	\label{fig:firstConn}
\end{figure}

La Figure \ref{fig:firstConn} permet d'expliquer la première connexion d'un utilisateur.
Alice veut s'enregistrer auprès du KGC, ainsi le KGC lui renvoie les paramètres publiques ($mpk_S$ et $mpk_E$) si aucun utilisateur n'a déjà cette adresse email (ID). Ces paramètres publiques sont assez lourds, en effet, ils font environ 52 kB au total.

L'utilisateur va alors créer sa valeur secrète puis générer sa clé publique.
Pour finir, Alice envoie sa clé publique au KGC afin qu'il l'associe à son ID et puisse le donner aux personnes qui veulent envoyer un mail à Alice.
\subsection{Envoi d'un message}
Présentation des différentes actions faites lors de l'envoi d'un message d'Alice à Bob. La Figure \ref{fig:aliceSends} permet d'avoir un aperçu du fonctionnement du protocole.
%TODO modifier schéma pour contact KGC au début ou récupération en local
\begin{figure}[h!]
	\centering
	\begin{sequencediagram}
		\newthread{A}{Alice}{}
		\newinst[7]{B}{KGC}{} 
		\newinst[2]{C}{Bob}{}
		\begin{call}{A}{Comm. with bob@mail.ch}{B}{$PKE_{Bob} $}
		\end{call}
		\postlevel
		\begin{call}{A}{Extract alice@mail.ch + time}{B}{$PPKS_{Alice}$}
		\end{call}
		\postlevel
		\begin{callself}{A}{$c'  = ENC_{PKE_{Bob}}(AES_K, bob@mail.ch + time)$}{}
		\end{callself}
		\postlevel
		\begin{callself}{A}{SetPrivSig $SKS_{Alice} = (PPKS_{Alice}, x)$}{}
		\end{callself}
		\postlevel
		\begin{callself}{A}{$s' = Sign(c', SKS_{Alice})$}{}
		\end{callself}
		\postlevel
		\begin{callself}{A}{$c, t = AESGCM_{AESK}(message)$}{}
		\end{callself}
		\postlevel
		\begin{call}{A}{time, c', c, t, s', IV}{C}{}
		\end{call}
	\end{sequencediagram}
	\caption{Alice envoie un message à Bob}
	\label{fig:aliceSends}
\end{figure}

Ainsi, l'envoi d'un message se déroule comme suit :
\begin{itemize}
	\item Tout d'abord, Alice va récupérer le clé publique de Bob via son ID (aka email).
	\item Elle devra aussi récupérer sa clé privée partielle de signature pour créer ses clés privées afin de signer le message. Elle va le faire à l'aide de son ID et du même timestamp qu'utilisé pour la suite.
	\item Elle va ensuite tirer une valeur aléatoire dans Gt qui représentera sa clé AES pour la suite. Elle va chiffrer cet élément à l'aide de la clé publique de Bob et de son ID complété par un timestamp. Ce timestamp sert à garder une certaine Forward Secrecy comme expliqué dans la Section \ref{subsec:pseudoSecrecy}. Le texte chiffré sera c'.
	\item Elle va calculer la signature du texte chiffré donné (s' sur la Figure \ref{fig:aliceSends})
	\item Alice utilisera un chiffrement authentifié comme AES\_GCM pour chiffrer et authentifier son mail à Bob, t pour le tag et c pour le texte chiffré.
	\item Finalement elle va envoyer tous ces éléments à Bob (à savoir, l'ID utilisé, c, c', t, s' et l'IV utilisé pour AES\_GCM).
\end{itemize}

\subsection{Réception d'un message}
Présentation des différentes actions faites lors de la réception d'un message pour Bob. La Figure \ref{fig:bobReceives} permet d'avoir un aperçu du fonctionnement de la réception.
\begin{figure}
[h!]
	\centering
	\begin{sequencediagram}
		\newthread{A}{Bob}{}
		\newinst[7]{B}{KGC}{} 
		\newinst[2]{C}{Alice}{}
		\begin{messcall}{C}{time, c', c, t, s', IV}{A}
		\end{messcall}
		\postlevel
		\begin{call}{A}{PK of alice@mail.ch}{B}{$PK_{Alice}$}
		\end{call}
		\postlevel
		\begin{callself}{A}{$s' == Verify(c', PK_{Alice})$}{}
		\end{callself}
		\postlevel
		\begin{call}{A}{Extract bob@mail.ch + time}{B}{$PPKE_{Bob}$}
		\end{call}
		\postlevel
		\begin{callself}{A}{SetPriv $SKE_{Bob}' = $}{}
		\end{callself}
		\postlevel
		\begin{callself}{A}{$AES_K = DEC_{SKE_{Bob}}(c', ID=bob@mail.ch+time)$}{}
		\end{callself}
		\postlevel
		\begin{callself}{A}{$message = AESGCM_{AES_K}(c,t, IV)$}{}
		\end{callself}
	\end{sequencediagram}
	\caption{Bob reçoit le message}
	\label{fig:bobReceives}
\end{figure}

Ainsi la réception déclenche les étapes suivantes :
\begin{itemize}
	\item A la réception, la première chose à faire est de vérifier le texte chiffré de la clé AES. Pour cela, on va demander la clé publique d'Alice au KGC. Puis on va vérifier ce texte chiffré c' à l'aide de sa signature s'.
	\item Ensuite Bob va récupérer sa clé privée partielle via le KGC en fournissant son ID avec le timestamp envoyé par Alice. Il va ainsi pouvoir former sa clé privée.
	\item Avec s clé privée il va pouvoir déchiffrer c' et obtenir la clé AES pour la suite.
	\item Une fois que l'on a la clé AES on peut simplement déchiffrer à l'aide d'AES\_GCM le chiffré c pour obtenir le message initial.
\end{itemize}
