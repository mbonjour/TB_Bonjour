\chapter{Fichiers}
\label{ch:fichiers}
Liste des fichiers annexes au rapport, ce qu'ils contiennent et comment les utiliser si besoin.
\section{Code du \textit{Proof Of Concept}}
Le code est en annexe du rapport avec le nom \textit{POCCertificatelessCryptography}.
Le \textit{README.md} présent à la source devrait être suffisant pour compiler soi-même le code. 

Le code est commenté à la manière \textit{Doxygen} afin d'avoir des informations à chaque fois qu'une fonction est référencée. De plus, le code est commenté afin de comprendre le déroulement global des différentes parties. Le code va générer ainsi trois exécutables dont le client, le serveur et un test afin de calculer le temps mis par les algorithmes et utilisé anciennement pour se représenter le déroulement global d'un échange entre Alice et Bob.
\section{Tableaux comparatifs}
Les tableaux comparatifs cités dans le chapitre \ref{ch:analysis} apparaissent sous forme de feuille dans un fichier excel se nommant \textit{ComparatifsCLPKCSchemes.xlsx}. 
La feuille nommée CLEs contient un comparatif des schémas de chiffrement tandis que la feuille CLSs contient un comparatif des schémas de signatures.

Ces tableaux sont mis en annexes car il représente une grande quantité de données qui apparaissait mal dans ce rapport.