\chapter{Outils utilisés pour la compilation}
\label{ch:outils}
\section{Sagemath}
Sagemath est un framework python qui permet de faire des calculs mathématiques simplement en quelques lignes de code. C'est très pratique pour tester certains calculs rapidement. Cependant, python étant un langage interprété c'est souvent plus lent que des implémentations faites avec des langages compilés. Dans ce projet il a surtout été utilisé afin de se familiariser avec le fonctionnement de la primitive choisie.

\section{RELIC Toolkit}
Pour pouvoir faire des calculs de \textit{Pairings} sur des courbes elliptiques je me suis fié à RELIC Toolkit~\cite{relic-toolkit} qui est une librairie C permettant ce genre de calculs assez simplement. Le choix de la librairie est expliquée plus en détails dans le chapitre \ref{ch:impl}.

Cette librairie demande à être compilée avec une certaine courbe et certaines options (typiquement fonction de hachage et autres...). Des presets existent et c'est donc ce que j'ai utilisé pour ce POC. Cela demande donc de fournir la librairie précompilée avec les bonnes options pour l'utilisateur. L'inconvénient c'est donc que pour mettre à jour une courbe il va falloir recompiler toute la librairie et la fournir à l'utilisateur.
\section{Libsodium}
Pour faire du chiffrement authentifié l'utilisation de la librairie  libsodium\footnote{https://libsodium.gitbook.io/doc/} était le premier choix. En effet, la librairie ne m'est pas totalement étrangère et est très complète au niveau du chiffrement authentifié, gestion de la mémoire sécurisée, algorithmes de dérivation de clés et d'autres. Ainsi le POC nécessite d'avoir libsodium installé sur la machine exécutant le client du POC.
\section{Libbinn}
Binn\footnote{\url{https://github.com/liteserver/binn}} est une librairie de sérialisation en C, permettant de créer des objets, des maps et des listes. Cela a été très utile pour la sérialisation des données entre le serveur et le client du POC et lors de l'enregistrement des différents paramètres du côté du client. Le POC utilise cette librairie et il faut donc avoir la librairie installée pour la compilation.
\section{Libetpan}
LibetPan\footnote{\url{https://www.etpan.org/libetpan.html}} est une librairie visant à simplifier la gestion des emails dans le code C. Ainsi cette librairie est utilisée pour la récupération des emails depuis le serveur GMail mais aussi pour \textit{parser} les mails reçus et analyser le contenu afin de pouvoir le déchiffrer. Cette librairie aurait pu être utilisée pour l'envoi des mails, mais la connaissance de cette librairie est parvenue après libcurl.
\section{Libcurl}
Libcurl\footnote{\url{https://curl.haxx.se/libcurl/}} permet d'envoyer toutes sortes de requêtes simplement à l'aide d'une API de programmation assez simple et efficace. Dans le POC elle est utilisée afin d'envoyer des requêtes SMTPs pour envoyer des emails chiffrés générés par le POC.
\section{UnQlite}
UnQlite\footnote{\url{https://unqlite.org/}} permet d'avoir une base de données NoSQL gérée en C et ainsi ne pas avoir de base de données complexes pour une utilisation simple comme dans ce cas. En effet, lors de l'enregistrement des clés publiques des utilisateurs, il suffit juste de pouvoir retrouver telle clé par rapport à tel ID. Ainsi une base de données avec ce genre d'entrée suffit. Sur leur site UnQlite propose une version précompilée de la librairie payante, cependant le github\footnote{\url{https://github.com/symisc/unqlite}} est publique et l'on peut donc le compiler de nous même. Ce qui doit d'ailleurs être fait dans le cadre de ce POC. En effet l'instance UnQlite fonctionnant sur le KGC et le KGC fonctionnant dans un environnement multi-threadé il est nécessaire de compiler la librairie avec un flag de compilation activant le support multiThread.