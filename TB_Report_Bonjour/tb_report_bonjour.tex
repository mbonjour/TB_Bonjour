\documentclass[a4paper,11pt,twoside,openright]{book} % Type du document

% compiler avec : pdflatex, bibtex, pdflatex, pdflatex


% +---------------------------------------------------------------+
% | Language
% +---------------------------------------------------------------+
\usepackage[T1]{fontenc}
\usepackage[utf8]{inputenc}
\usepackage[french]{babel}
\usepackage{url}
\usepackage{pgf-umlsd}
\usepackage{adjustbox}
\raggedbottom


\newif\ifisconfidential	\isconfidentialfalse

\newif\ifisdraft\isdraftfalse



% +---------------------------------------------------------------+
% | Paramètres
% +---------------------------------------------------------------+

\newcommand{\TBtitle}{Chiffrement/Signature d'Emails}
\newcommand{\TBsubtitle}{ }%laisser vide si pas de sous-titre
\newcommand{\TByear}{2020}
\newcommand{\TBacademicYears}{2019-2020}

\newcommand{\TBdpt}{Département TIC}
\newcommand{\TBfiliere}{Filière Télécommunications}
\newcommand{\TBorient}{Orientation Sécurité de l'information}

\newcommand{\TBauthor}{Mickael Bonjour}
\newcommand{\TBsupervisor}{Prof. Alexandre Duc }
\newcommand{\TBindustryContact}{}
\newcommand{\TBindustryName}{}
\newcommand{\TBindustryAddress}{%
  Rue XY\\
  1400 Yverdon-les-Bains
}

% Confidentiel?
% uncomment if confidential / comment if not confiential
% \isconfidentialtrue

\newcommand{\TBresumePubliable}{
Dans ce travail de bachelor je vais analyser les besoins classiques d'un système de messagerie éléctronique. De plus, une analyse des solutions actuelles de messagerie sécurisée est faite afin d'identifier les propriétés cryptographiques mise en avant dans de tels systèmes. De ces propriétés je détermine une primitive cryptographique adéquate au problème et qui permettrait d'avoir les propriétés cryptographiques mentionées auparavant. Ensuite j'établirais un \textit{Proof Of Concept} amenant la technologie choisie dans un cadre de messagerie électronique sécurisée. Ce Proof of Concept se base sur la Certificateless Public Key Cryptography qui a été la primitive choisie grâce à ses propriétés intéressantes dans un contexte de mail sécurisés. J'évalue ma solution par rapport à d'autres solutions présentent sur le marché au niveau du temps de chifrement/déchiffrement, de l'overhead introduit et des propriétés cryptographiques des différents systèmes. 
}

% +---------------------------------------------------------------+
\setlength{\emergencystretch}{100pt}
% Not reccommended but avoid text overfull

% +-[set path]-------------------------------------+
\usepackage{template/TB-style}
\usepackage{template/TB-macros}
\usepackage{template/TB-template}
%\graphicspath{images/}

% TODO : Vérifier si ok, ou mettre quand même 
\setcounter{tocdepth}{1}

\begin{document}

\frontmatter
\pagestyle{empty}

% TITLE and template
% +---------------------------------------------------------------+

\TBmaketitle

\pagestyle{frontmatter}

\TBsecondTitle

\TBpreambule

\TBauthentification


% Cahier des charges
% +---------------------------------------------------------------+
\chapter{Cahier des charges}


\section*{Résumé du problème}
Les outils de chiffrement et de signature d'email actuels se résument principalement à S/MIME et à PGP. 

Ces deux solutions sont anciennes, souffrent assez régulièrement de nouvelles vulnérabilités et ne proposent pas certaines propriétés cryptographiques qui pourraient être utiles (par exemple, la "forward secrecy". Le but de ce travail de bachelor est d'étudier quelles propriétés seraient utiles pour la sécurisation des emails, de proposer un nouveau protocole les implémentant et de développer un proof of concept. 
\subsection*{Problématique}
La problématique principale est résumée ci-dessus mais le principal problème c'est surtout que les technologies utilisées sont vieilles et elles souffrent de vulnérabilités par conception qui ont été mitigées en enlevant des options à l'utilisateur.
\subsection*{Solutions existantes}
%TODO: Compléter/Spécifier !
PGP, S/MIME, PEP, messagerie instantanée (Signal)...
\subsection*{Solutions possibles}
Une solution possible est d'utiliser le système mis en place dans ce travail, cependant il faudrait une relecture et des analyses plus approfondies pour s'en assurer.
Un des points possibles c'est de passer plus à de la messagerie instantanée dans la mesure du possible.
Ou encore de s'orienter sur des nouvelles technologies comme PEP ou PGP mais en appliquant strictement les \textit{Best Practices} et en se formant un peu sur leur utilisation qui n'est pas donnée à tout le monde.
\section*{Cahier des charges}
Voici un résumé du cahier des charges sous formes d'objectifs à atteindre :
\begin{itemize}
	\item Analyser les besoins d’un système d’E-mails actuel.
	\item Analyser et étudier les solutions de sécurité existantes.
	\item Comprendre et évaluer les propriétés cryptographiques défendues.
	\item Établir une liste des propriétés cryptographiques voulues pour un système de mails sécurisés.
	\item Trouver une primitive cryptographique satisfaisant les besoins énoncés et l’étudier pour en comprendre les bases et les besoins nécessaires en termes de sécurité.
	\item Établir la spécification pour un nouveau protocole en utilisant la primitive choisie.
	\item Faire un Proof Of Concept du protocole proposé.
\end{itemize}

Si le temps le permet: 
\begin{itemize}
	\item Comprendre plus en détails les mathématiques derrière la primitive utilisée.
	\item Faire un prototype de client mail utilisant une architecture mise en place pour le POC.
\end{itemize}

%\subsection*{Objectifs}

\subsection*{Déroulement}
Tout d'abord je vais m'intéresser à faire une évaluation des concepts existants en messagerie sécurisée, tel que PGP et S/MIME pour les emails ou encore Signal pour la messagerie instantanée. Ayant vu ce qu'il se fait j'essaie de trouver une solution alternative pour le chiffrement et la signature d'emails. De là je vais conceptualiser un protocole et l'implémenter au sein d'un \textit{Proof Of Concept}.
\subsection*{Livrables}
Les délivrables seront les suivants :
\begin{enumerate}
\item Une documentation contenant :
	\begin{itemize}
	\item Une analyse de l'état de l'art
	\item La décision qui découle de l’analyse
	\item Spécifications
	\item L'implémentation faites et les choix faits
	\item Proof Of Concept
	\item Les problèmes connus
	\end{itemize}
\item Le code du \textit{Proof Of Concept} fait, expliqué à l'aide de commentaires.
\end{enumerate}




% TOC
% +---------------------------------------------------------------+
\tableofcontents
\clearpage


% Content
% +---------------------------------------------------------------+

\mainmatter
\pagestyle{plain}

\chapter{Introduction}
\label{ch:intro}

Ce travail de Bachelor a pour but de sensibiliser à la vulnérabilité dans les systèmes actuels de messagerie électronique. Il propose aussi un nouveau protocole permettant de sécuriser ce type de messagerie à l'aide d'une primitive cryptographique peu implémentées, le \textit{Certificateless Public Key Cryptography}. Ma démarche ans ce travail de bachelor est de voir si des solutions s'offrent à nous en considérons ce qu'il se fait sur le marché actuellement. Et en essayant d'améliorer les solutions actuelles proposées qui peuvent souffrir d'un manque de sécurité assez souvent ou (et plus souvent) un manque de simplicité d'utilisation.\\
Ce travail est découpé en plusieurs parties. En effet, on commence par une analyse de l'état de l'art, donc à regarder ce qui existe et voir pourquoi il faudrait de nouvelles solutions. Puis une présentation de la primitive cryptographique utilisée pour ma proposition dans ce travail. Enfin la présentation de l'architecture de mon protocole et une implémentation proposée en \textit{Proof Of Concept} ainsi que les choix importants qui ont été faits en rapport à cette implémentation.

\chapter{Analyse - État de l'art}
\label{ch:analysis}
%todo utilisation emails globales
\noindent La problématique principale est la difficulté d’utilisation des sécurités mises en places au-dessus des protocoles de base pour les emails. 
De plus des vulnérabilités (EFAIL en \ref{attacks:EFAIL}) qui réussissent à récupérer le texte clair a démontré que la sécurité n’était pas bien implémentée. Les vulnérabilités proviennent plus d'un défaut de conception inhérent aux mails.
Je vais ici décrire les principaux problèmes trouvés sur PGP et S/MIME lors de mes tests d’utilisation et ce que j’ai trouvé durant mes recherches. De plus je vais analyser des systèmes de mails sécurisés tel que Protonmail et Tutanota.
\section{Protocoles existants}
Ici je vais regarder les protocole existants afin de mettre en place des liaisons sécurisées de messagerie éléctronique.
%todo ajouter fonctions techniques - choix des users des suites crypto
\subsection{PGP}
\paragraph*{Fonctionnement.}
PGP (Pretty Good Privacy ou Assez bonne confidentialité) est un moyen de chiffrer des données (mails, fichiers, …) qui est beaucoup représenté lorsque l’on parle de sécurité email car c’est le plus utilisé avec S/MIME (c.f. \ref{protocols:SMIME}). C’est une méthode de chiffrement hybride (utilise le chiffrement symétrique et assymétrique) qui fonctionne comme montré sur la Figure \ref{fig:PGP_101}.

\begin{figure}[h!]
\includegraphics[width=10cm]{images/PGP_101.png}
\centering
\caption{Le fonctionnement global de PGP}
\label{fig:PGP_101}
\end{figure}

\noindent Ce fonctionnement hybride est défendu à cause de la lenteur et la non-praticité d’un chiffrement asymétrique sur un certain nombre de données. Ainsi en chiffrant uniquement la clé symétrique qui a servi à chiffrer le tout l’on peut déchiffrer bien plus rapidement et simplement le message (typiquement avec un chiffrement symétrique tel qu’AES qui a le droit à des instructions dédiées dans certains processeurs). Contrairement à des chiffrements asymétriques qui sont plus contraignants. Et l'on n'utilise pas directement le chiffrement symétrique car il a besoin d'un secret partagé dès le début de la communication.
PGP utilise un système de clés… PGP est aussi critiqué pour son manque de "Forward Secrecy"...
\paragraph*{Propriétés cryptographiques.}
Le problème qui est souvent reproché à PGP c'est qu'il n'implémentes pas de \textit{Forward Secrecy}. La \textit{Forward Secrecy} permet d'affirmer que si l'on a une brèche à un instant \textit{T}, et qu'un attaquant récupère cette clé, il ne pourra pas déchiffrer les anciens messages.
De plus, la gestion des clés PGP est très problématique, en effet lors de mes tests il était difficile de connecter un serveur de clés p. ex. Ou de recevoir une clé d'un correspondant pour la sauvegarder. Et même en la recevant, comment savoir si cette clé n'a pas été modifiée via un \textit{MITM} p.ex. ? -> utiliser un autre canal pour vérifier l'empreinte.
\paragraph*{Web of Trust.}
Comment faire confiance a une clé -> surtout pour email -> Comment initialiser une confiance ?
\paragraph*{Autocrypt.}
Autocrypt est une manière d'échanger des clés entres emails, ces échanges ne sont pas considérés sécurisé par la communauté (Wikipedia -> à creuser). C'est une façon de s'échanger des clés de manière automatisée mais pas forcément sécurisée (utilise les mails).
\paragraph*{Utilisation.}
Pour mes tests j’ai fait en sorte de trouver l’utilisation la plus simple possible pour voir si un utilisateur lambda pouvait arriver à mettre en place ce genre de sécurité. Il s’est avéré que cela était assez simple au départ, mais dès lors que l'on veut envoyer un mail chiffré à un correspondant cela se complique un peu. J’ai juste eu à installer un Add-On sur mon logiciel de messagerie (Thunderbird dans mon cas) qui s’appelle Enigmail. Ensuite Enigmail a générer mes clés PGP (de manière totallement opaque -> à creuser). Puis j’ai écrit un mail, en appuyant sur un petit cadenas mon mail partait chiffré et signé (uniquement si on a la clé du correspondant). Bien, cependant c’est très opaque et on ne sait pas ce qu'Enigmail et Autocrypt font réellement derrière les décors. L’utilisateur doit encore choisir s’il veut chiffrer ses mails ou non par contre il faut que le destinataire utilise PGP et que l’on ait sa clé publique. 
J’ai donc expérimenté à plus bas niveau ce qu’il se passait.
%TODO Aller dans les détails
\subsection{PEP}
\paragraph*{Citation.}
\textit{By default, communications between pep peers always work end-to-end encrypted – no eavesdrop-ping in between shall be possible by design.}
\paragraph*{Utilisation.}
pep assure un chiffrement de bout-en-bout par design, ils n'ont en effet pas de serveurs en soit et chiffre à l'aide d'un \textit{handshake} fait entres les deux personnes via des \textit{trustwords}. Ce sont des mots qu'il faut vérifier entres les deux partis afin d'être sûr que la connexion est bien authentifiée.
\begin{figure}[h!]
    \includegraphics[width=15cm]{images/conceptualpEp.png}
    \centering
    \caption{Le fonctionnement global de pEp}
    \label{fig:PEP_global}
\end{figure}
-> à creuser mais à priori PEP utilise PGP pour le chiffrement des messages.
\subsection{S/MIME}
\label{protocols:SMIME}
\paragraph*{Fonctionnement.}
Basé sur le même principe que PGP principalement, mais avec des certificats pour prouver la légitimité des clés publiques. Pour l'utilisation il faut se créer un certificat, plusieurs classes de confiance existe. 
\paragraph*{Propriétés cryptographiques.}
\paragraph*{Utilisation.}
\section{Implémentations existantes}
\subsection{Protonmail}
\paragraph*{Revendications.}
Protonmail revendique beaucoup de propriétés cryptographiques, tel que le zero-access encryption. Et l’end-to-end chiffrement + zero-knowledge pour les messages sécurisés, même avec leur fonctionnalité de (Chiffrement vers l'extérieur) utilisant AES256-GCM. 
Pour l'authentification Protonmail utilise une manière fortement sécurisée (SRP) pour ne pas avoir d'informations direct sur le mot de passe de l'utilisateur.
\paragraph*{Fonctionnement.}
Protonmail a plusieurs modes de fonctionnement dépendant du destinataire final. En effet de Protonmail à Protonmail les mails sont chiffrés à l'aide de PGP automatiquement. L'on peut utiliser Protonmail pour utiliser PGP si l'on a la clé de notre destinataire par exemple. Et l'on peut écrire un mail chiffré à quelqu'un qui n'utilise pas PGP grâce à une fonctionnalité de chiffrement vers l'extérieur.
Cette fonctionnalité enverra une URL au destinataire qui, en la consultant, pourra déchiffrer le mail en utilisant un mot de passe communiqué de manière sécurisées entres les deux partis auparavant.
\paragraph*{Open Source.}
Tout leur code est open-source afin d'avoir une validation externe, de plus ils ont un programme de Bug Bounty pour les chercheurs.
\subsection{Tutanota}
\paragraph*{Fonctionnement.}
Tout ce que j'ai vu pour le moment c'est que Tutanota utilise AES128-CBC ? Mais dans PGP ou ailleurs ?
%\subsection{Bitmessage}
\section{Attaque existantes}
\subsection{Défauts webmail}
Selon un chercheur~\cite{DBLP:journals/iacr/Kobeissi18a} l'infrastructure de Protonmail aurait des failles via son webmail. Mais son papier est en fait plus général et parle des webmails en règle général.
Il part du principe que les serveurs de Protonmail ne sont pas des serveurs à faire confiance, pour ainsi prouver le zero-knowledge de Protonmail. Par contre, le fait qu'il ne peuvent pas être mis en confiance est un problème selon lui, car c'est ces serveurs qui vont délivrer le code d'OpenPGP afin de faire le chiffrement. 
Cela indique que si Protonmail était corrompu le fait d'avoir le code délivré par Protonmail pourrait avoir des effets néfastes. Comme p.ex l'extraction de la clé privée PGP. La conclusion est que dès le moment où vous avez utilisé une fois le webmail de protonmail la clé PGP est corrompue.
\subsection{EFAIL}
\label{attacks:EFAIL}
Malgré ces sécurités qui pourraient être mises en place à l’heure actuelle, une attaque nommée EFAIL~\cite{DBLP:conf/uss/PoddebniakD0ISF18} a été faite en 2018 et est toujours possible aujourd’hui(à vérifier / tester). En effet cette attaque a seulement été mitigée en évitant d’afficher les contenus HTML et les images dans boites mails de base. Car le problème vient de là principalement, des problèmes sont liés aussi aux modes de chiffrement utilisé (typiquement CBC et CFB) grâce à des "gadgets".
Cette attaque permet en fait d'injecter une image dans l'HTML du message (typiquement dans les headers du mail), puis faire en sorte de récupérer le contenu du message déchiffrer dans un paramètre de l'URL.

\section{Signal}
L'analyse s'est faite aussi pour la messagerie instantanée à cause de sa ressemblance avec la messagerie électronique. 
%Cela m'a finalement permis de me rendre compte que ce n'était pas des protocoles idéaux.
%TODO compléter l'analyse sur Signal
\subsection{Fonctionnement}
%TODO commenter la figure
\begin{figure}[h!]
	\centering
	\includegraphics[width=8cm]{images/signalFonctionnement.png}
	\caption{Schéma fonctionnement de Signal\cite{doubleratchet}}
	\label{fig:signal}
\end{figure}
\subsection{Problèmes d'intégrations}
Le problème avec le protocole Signal quant à mes besoins niveaux mails est la \textit{forward secrecy} qui est très fort. En effet comme vu dans le chapitre précédent il utilise une clé par message grâce au \textit{Double Ratchet}. Cependant ce fonctionnement comporte un gros problème en rapport aux mails, en effet si l'on veut pouvoir récupérer les anciens mails reçus/envoyés cela devient vraiment compliqué. En effet, la \textit{forward secrecy} est une propriété utile dans un système de mail, mais faut pouvoir aussi récupérer les messages facilement si l'on connait la clé privée.
\section{Compromis}
Pour passer à l'implémentation concrète d'un nouveau protocole il faut faire des compromis et aller chercher dans des primitives moins connues.\\
Je suis tout de même rester sur un système de clés publiques comme PGP le fait. Cependant cette primitive a une identité propre à chaque clé publique.\\
De plus pour avoir une \textit{forward secrecy} l'on peut ajouter une notion de temps ou de token à l'ID pour chaque batch de messages.
\subsection{Résultats des recherches}
Comme mentionné avant les recherches ont beaucoup été orientées sur le protocole Signal qui a une très bonne forward secrecy, résilience et break-in recovery. Cependant le problème avec l'utilisation des mails c'est d'avoir envie de consulter tout ces mails depuis n'importe quel appareil. Ce n'est malheureusement pas le cas avec Signal à moins de conserver une \textit{root key} quelque part qui ferait s'effondrer les caractéristiques principales du protocole.\\
%TODO : Refaire ce paragraphe
S/MIME est la solution prédominante pour s'envoyer des mails chiffrés cependant il est compliquer de l'utiliser. En faisant quelques essais de mon côté je me suis heurté à beaucoup de difficultés et de problèmes avec les clés PGP.

\section{Primitives}
\subsection{Primitives analysées}
%TODO : Expliquer les différentes primitives
- Certificateless PKC~\cite{DBLP:conf/asiacrypt/Al-RiyamiP03}\\
- HIBE - Hierarchised Based encryption\\
- Identity based encryption
%TODO à compléter
\subsection{Primitive choisie}
- Certificateless PKC~\cite{DBLP:conf/asiacrypt/Al-RiyamiP03}\\
Tout d'abord le HIBE aurait été compliqué à mettre en place aussi à cause de sa forte forward secrecy, la hiérarchie doit de plus être commune sur tout les appareils, ce qui prendrait probablement du temps à un nouvel arrivage d'appareil pour un compte donné.
J'ai choisi cette primitive au final car elle proposait des propriétés très intéressantes pour ma manière d'implémenter dans les mails cela. En effet, similaire à de l'identity based avec un ID pour désigner une clé publique. Le problème avec l'identity based encryption c'est le fait que le serveur central génère la clé publique et la clé secrète de l'utilisateur, cela amène ce qu'on appelle le \textit{key escrow} problème. C'est le fait qu'une entité connaisse à elle seule toutes les clés. Ce problème est résolu dans le certificateless en introduisant des \textit{Partial Private Keys} permettant d'avoir une clé secrète partiellement générée par le serveur et par l'utilisateur.
\section{Recherches sur la primitive}
\subsection{Principes mathématiques}
Les variantes de \textit{Certificateless Cryptography} choisies utilisent un concept appelé les \textit{pairings} ou \textit{bilinear map groups}.
Des groupes tels que $\mathbb{G}_1, \mathbb{G}_2, \mathbb{G}_T$ d'un ordre premier \textit{p} pour lesquels il existe un mapping $e : \mathbb{G}_1 \times \mathbb{G}_2 \rightarrow \mathbb{G}_T$ avec les propriétés suivantes :\\
1. Bilinéarité : $e(g^a, h^b) = e(g, h)^{ab}$ pour tout $(g,h) \in \mathbb{G}_1 \times \mathbb{G}_2$ et $a,b \in \mathbb{Z}$;\\
2. Pas de dégénérescence : $e(g,h) \neq 1_{\mathbb{G}_T} $ tant que $g,h \neq 1_{\mathbb{G}_{1,2}}$;\\
%TODO : Détails mathématiques
\subsection{Schémas Certificateless de Chiffrement}
Pour choisir parmi les nombreux schémas existants en certificateless pour le chiffrement j'ai établi un tableau comparatif des différentes manières de faire, inspiré de~\cite{bookIntroCertificateless}. En suivant ce tableau je me suis rendu compte que la construction de Dent-Libert-Paterson~\cite{DBLP:conf/pkc/DentLP08} était probablement la plus adaptée en vue des propriétés qu'elle présentait. Le tableau se trouve en annexe \ref{ch:fichiers}.
%TODO Expliquer les propriétés
\subsection{Détails techniques}
Les détails techniques sur le chiffrement avec la \textit{Certificateless Cryptography}.
Le chiffrement se base sur le problème difficile \textit{The Decision 3-Party Diffie-Hellman Problem} (3-DDH). \\C'est de décider si $T =g^{abc} ayant (g^a, g^b, g^c, T) \in \mathbb{G}_4$.\\
Pour expliquer les détails techniques je vais ici montrer les calculs faits dans le schéma choisi~\cite{DBLP:conf/pkc/DentLP08} et les expliquer :\\
\textbf{Setup($1^k, n$) :} Avec $\mathbb{G}_1, \mathbb{G}_2, \mathbb{G}_T$ avec un ordre $p > 2^k$. $g$ est un générateur de $\mathbb{G}_1$. Ensuite  $g_1 = g^\gamma$ pour un $\gamma \leftarrow  \mathbb{Z}_p^*$ aléatoire. Puis $g_2 \leftarrow \mathbb{G}_2$. Deux vecteurs (U,V) seront tirés aléatoirement dans $\mathbb{G}_2^{n+1}$ en tant que fonctions de hash notés :
\[F_u(ID) = \]
L'on va aussi prendre une fonction de hash résistante aux collisions : $H : \{0,1\}^* \rightarrow \{0,1\}^n$. Au final notre $mpk$ (master public key) est :
\[mpk \leftarrow (g, g_1, g_2, U, V)\]
Et le $msk$ (master seret key) est $msk \leftarrow g_2^\gamma$.
\\
\textbf{Extract($mpk, \gamma, ID$) :} On prend $r \leftarrow \mathbb{Z}_p^*$ puis on retourne $d_{ID} \leftarrow (d_1, d_2) = (g_2^\gamma * F_u(ID)^r, g^r)$\\
\textbf{SetSec($mpk$) :} Retourne un secret aléatoirement choisi $x_{ID} \leftarrow \mathbb{Z}_p^*$.\\
\textbf{SetPub($x_{ID}, mpk$) :} Retourne $pk_{ID} \leftarrow (X,Y) = (g^{x_ID}, g_1^{x_ID})$.\\
\textbf{SetPriv($x_{ID}, d_{ID}, mpk$) :} On choisit $r' \leftarrow \mathbb{Z}_p^*$ puis on reprends $(d_1, d_2) \leftarrow d_{ID}$ et l'on va prendre en secret key : 
%TODO : Revoir si assez complet :
\[sk_{ID} \leftarrow (s_1, s_2) = (d_1^{x_{ID}} * F_u(ID)^{r'}, d_2^{x_{ID}} * g^{r'})\]
Avec $sk_{ID}$ étant la clé secréte de l'utilisateur, donnée par l'Extract (notre Partial Private Key) et la valeur secrète de SetSec.\\
\textbf{Encrypt :}\\
\textbf{Decrypt :}
\subsection{Schémas Certificateless de Signature}
%TODO : Expliquer Malicious KGC et mettre tableaux
Pour choisir parmi les nombreux schémas certificateless pour la signature j'ai établi un tableau comparatif des différentes manières de faire inspiré de~\cite{bookIntroCertificateless}. En analysant les différentes possibilités dans ce tableau il y a peu de solutions se dégage, en effet l'on peut voir que beaucoup de schémas de signature sont cassés, mon choix s'est porté au final sur la construction de Zhang et Zhang~\cite{DBLP:conf/icc/ZhangZ08a} pour des signatures robustes en Certificateless. J'ai pris cette construction car elle est résistante au Malicious KGC (si le KGC a été setup avec des paramètres vulnérables) et en plus elle date de 2008 et n'a apparemment pas été cassée. Le tableau se trouve en annexe \ref{ch:fichiers}.
\subsection{Détails techniques}
Les détails techniques sur la signature avec la \textit{Certificateless Cryptography}.
La signature se base sur le problème difficile \textit{The ComputationalDiffie-Hellman Problem} (CDH). \\Ayant $P, aP, bP$ où $a,b$ aléatoires $\in \mathbb{Z}_q^*$ il n'est pas possible de trouver $abP$.\\
Pour expliquer les détails techniques je vais ici montrer les calculs faits dans le schéma choisi~\cite{DBLP:conf/icc/ZhangZ08a} et les expliquer :\\
\textbf{Setup :}\\
\textbf{Extract :}\\
\textbf{SetSec :}\\
\textbf{SetPub :}\\
\textbf{SetPriv :}\\
\textbf{Sign :}\\
\textbf{Verify :}


\chapter{Architecture / Design du protocole}
% TODO Spécifications en texte via chapitre ou autre plus précise et ajouter un chapitre sur les notations
\label{ch:arch}
Dans ce chapitre, je vais m'intéresser à expliquer le fonctionnement de la \textit{certificateless cryptography} et démontrer comment je l'ai utilisée afin de l'intégrer à un protocole de chiffrement de mail.
\section{Architecture globale}
Dans cette Figure \ref{fig:globalProtocol}, je présente uniquement l'architecture globale pour bien représenter les différents acteurs présents dans le protocole et ainsi avoir une vue d'ensemble pour faciliter la compréhension.
\begin{figure}[h!]
	\centering
	\includegraphics[width=14cm]{images/SchemaGlobal.png}
	\caption{Schéma global du protocole}
	\label{fig:globalProtocol}
\end{figure}
%TODO compléter ce chapitre et préciser les actions des acteurs plus en détails
\section{Acteurs}
Les parties impliquées sont les suivantes comme vu à la figure \ref{fig:globalProtocol}.
\begin{itemize}
	\item Alice : L'envoyeur du mail en direction de Bob. Alice doit discuter avec le KGC pour construire sa clé privée (afin de signer) et récupérer la clé publique de Bob.
	\item Bob : Le destinataire du message, communique uniquement avec le KGC en ayant reçu le message d'Alice afin de récupérer sa clé publique pour vérifier la signature et de construire sa clé privé pour déchiffrer le message.
	\item KGC : Permet aux différents acteurs de pouvoir récupérer les clés publiques des clients, mais aussi de  recevoir les \textit{Partial Private Keys} qui permettent aux acteurs de construire leur clé privée. 
\end{itemize}
Ces partis sont les principaux présents dans un exemple de \textit{Certificateless Cryptography} dans mon système de mail.
\section{Fonctionnement Certificateless PKC}
Je vais ici découper les différents algorithmes présent dans le certificateless public key cryptography. En passant par le chiffrement et la signature.
Ces algorithmes seront accompagnés d'explications sur leur utilité. Les noms donnés aux algorithmes seront réutilisés ensuite pour les schémas afin de démontrer l'architecture du protocole mis en place. L'on peut voir des définitions spécifiques dans l'article sur lequel je me suis appuyé pour ce travail~\cite{DBLP:conf/pkc/DentLP08}.
\subsection{Chiffrement}
Liste des différents algorithmes de \textit{Certificateless Cryptography} et leur description, les détails techniques de leurs implémentations sont disponibles à la Section \ref{sec:primitiveSearch}.
\begin{itemize}
	\item \textit{Setup.} (seulement une fois par le KGC).
	\item \textit{Partial-Private-Key-Extract.} Calcul d'une clé privée partielles lorsque qu'un client le demande pour identité donnée.
	\item \textit{Set-Secret-Value.} Le client ne le fait qu'une fois pour tirer sa valeur secrète.
	\item \textit{Set-Private-Key.} Le client combine ses clés partielles et sa clé secrète pour obtenir une clé privée afin de déchiffrer les message reçus, chiffrés avec une certaine identité.
	\item \textit{Set-Public-Key.} Le client ne le fait qu'une fois, il calcule sa clé publique en fonction de sa valeur secrète.
	\item \textit{Encrypt.} Chiffre un message avec la clé publique du destinataire et son identité.
	\item \textit{Decrypt.} Déchiffre un message utilisant sa clé privée et l'identité utilisée pendant le chiffrement.
\end{itemize}
\subsection{Signature}
Pour la signature les algorithmes sont les mêmes avec une différence dans leur conception et évidemment le \textit{Encrypt} et \textit{Decrypt} sont remplacé par \textit{Sign} et \textit{Verify}.
%TODO compléter voir tableau
Dans la littérature certificateless les schémas de signatures sont beaucoup plus cassés que ceux de chiffrement apparemment (voir tableau ). Il faut donc faire attention à suivre les schémas afin de vérifier que le schéma choisi ne soit pas mis à mal.
\section{Design du protocole}
%TODO : ajouter setup
\begin{figure}
[h!]
	\centering
	\begin{sequencediagram}
		\newthread{A}{Alice}{}
		\newinst[8]{B}{KGC}{} 
		\begin{call}{A}{Initialisation with alice@mail.ch}{B}{OK, $mpk_E, mpk_S$}
		\end{call}
	\postlevel
		\begin{callself}{A}{\shortstack{SetSec $x_E = Z_p^*$ \\ SetSecSig $x_S = Z_p^*$}}{}
		\end{callself}
	\postlevel
		\begin{callself}{A}{\shortstack{SetPub $PKE_{Alice} = (g^{x_E}, g_{1}^{x_E})$\\SetPubSig $PKS_{Alice} =x_SP$}}{}
		\end{callself}
	\postlevel
		\begin{call}{A}{$PKE_{Alice}, PKS_{Alice}$}{B}{}
		\end{call}
		
	\end{sequencediagram}
	\caption{Schéma de la première connexion}
	\label{fig:firstConn}
\end{figure}

Dans la Figure \ref{fig:firstConn} l'on voit la première connexion d'un utilisateur.
Alice veut s'enregistrer auprès du KGC, ainsi le KGC lui renvoi les paramètres publiques (mpkS et mpkE) si aucun utilisateur n'a déjà cet email.
L'utilisateur va alors crée sa valeur secrète tirée aléatoire modulo p puis générer sa clé publique.
Pour finir Alice envoi sa clé publique au KGC afin qu'il l'associe à son ID et puisses le donner aux personnes qui veulent envoyer un mail à Alice.\\

%\begin{figure}[h!]
	%\centering
	%\includegraphics[width=14cm]{images/aliceSendsToBob.png}
	%\caption{Alice envoi un message à Bob}
	%\label{fig:aliceSends}
%\end{figure}

\begin{figure}[h!]
	\centering
	\begin{sequencediagram}
		\newthread{A}{Alice}{}
		\newinst[7]{B}{KGC}{} 
		\newinst[2]{C}{Bob}{}
		\begin{call}{A}{Comm. with bob@mail.ch}{B}{$PKE_{Bob} $}
		\end{call}
		\postlevel
		\begin{call}{A}{Extract alice@mail.ch + time}{B}{$PPKS_{Alice}$}
		\end{call}
		\postlevel
		\begin{callself}{A}{$c'  = ENC_{PKE_{Bob}}(AES_K, bob@mail.ch + time)$}{}
		\end{callself}
		\postlevel
		\begin{callself}{A}{SetPrivSig $SKS_{Alice} = (PPKS_{Alice}, x)$}{}
		\end{callself}
		\postlevel
		\begin{callself}{A}{$s' = Sign(c', SKS_{Alice})$}{}
		\end{callself}
		\postlevel
		\begin{callself}{A}{$c, t = AESGCM_{AESK}(mesage)$}{}
		\end{callself}
		\postlevel
		\begin{call}{A}{time, c', c, t, s', IV}{C}{}
		\end{call}
	\end{sequencediagram}
	\caption{Alice envoi un message à Bob}
	\label{fig:aliceSends}
\end{figure}

Dans la Figure \ref{fig:aliceSends} l'on voit comment se déroulerait l'envoi d'un message à Bob : 
\begin{itemize}
	\item Tout d'abord, Alice va récupérer le clé publique de Bob via son ID (aka email).
	\item Elle devra aussi récupérer sa clé privée partielle de signature pour créer ses clés privées afin de signer le message. Elle va le faire à l'aide de son ID et du même timestamp qu'utilisé pour la suite.
	\item Elle va ensuite tirer une valeur aléatoire dans Gt qui représentera sa clé AES pour la suite, elle va chiffrer cet élément à l'aide de la clé publique de Bob et de son ID complété par un timestamp. Ce timestamp sert à garder une certaine Forward Secrecy. Le cipher sera c'.
	\item Elle va calculer la signature du cipher donné (s' sur la figure)
	\item Alice utilisera un chiffrement authentifié comme AES\_GCM pour chiffrer et authentifié son mail à Bob, t pour le tag et c pour le cipher.
	\item Finalement elle va envoyer tout ces éléments à bob (à savoir, l'ID utilisé, c, c', t, s' et l'IV utilisé pour AES\_GCM).
\end{itemize}

\begin{figure}
[h!]
	\centering
	\begin{sequencediagram}
		\newthread{A}{Bob}{}
		\newinst[7]{B}{KGC}{} 
		\newinst[2]{C}{Alice}{}
		\begin{messcall}{C}{time, c', c, t, s', IV}{A}
		\end{messcall}
		\postlevel
		\begin{call}{A}{PK of alice@mail.ch}{B}{$PK_{Alice}$}
		\end{call}
		\postlevel
		\begin{callself}{A}{$s' == Verify(c', PK_{Alice})$}{}
		\end{callself}
		\postlevel
		\begin{call}{A}{Extract bob@mail.ch + time}{B}{$PPKE_{Bob}$}
		\end{call}
		\postlevel
		\begin{callself}{A}{SetPriv $SKE_{Bob}' = $}{}
		\end{callself}
		\postlevel
		\begin{callself}{A}{$AES_K = DEC_{SKE_{Bob}}(c', ID=bob@mail.ch+time)$}{}
		\end{callself}
		\postlevel
		\begin{callself}{A}{$message = AESGCM_{AES_K}(c,t, IV)$}{}
		\end{callself}
	\end{sequencediagram}
	\caption{Bob reçoit le message}
	\label{fig:bobReceives}
\end{figure}

Mais dans la Figure \ref{fig:bobReceives} l'on voit comment la réception du côté de Bob se déroulerait :
\begin{itemize}
	\item A la réception la première chose à faire et de vérifier le cipher de la clé AES. Pour cela l'on va demander la clé publique d'Alice au KGC. Puis on va vérifier ce cipher c' à l'aide de sa signature s'.
	\item Ensuite Bob va récupérer sa clé privée partielle via le KGC en fournissant son ID avec le timestamp envoyé par Alice. Il va ainsi pouvoir former sa clé privée.
	\item Avec s clé privée il va pouvoir déchiffrer c' et obtenir la clé AES pour la suite.
	\item Une fois que l'on a la clé AES l'on peut simplement déchiffrer à l'aide d'AES\_GCM c pour obtenir le message initial.
\end{itemize}


\chapter{Implémentation}
\label{ch:impl}

\section{Choix d'implémentations}
\subsection{Langage}
Au départ le choix du langage s'est porté sur sagemath (framework python) afin de mieux comprendre les différents calculs et faire un premier POC du chiffrement.
Cependant l'implémentation du POC était lente et le changement d'algorithme pour les pairings était difficile.
Je me suis donc orienté sur le C pour avoir de meilleures performances et pouvoir mieux gérer ma mémoire car c'est un point important dans des logiciels implémentant de la cryptographie. Pour pouvoir faire facilement des calculs sur les courbes elliptiques et les pairings en C il me fallait une librairie.

\begin{table}[h!]
	\centering
	\begin{tabular}{ |p{3cm}||p{3cm}|p{3cm}| }
		\hline
		\multicolumn{3}{|c|}{Temps des algorithmes entres langages [s]} \\
		\hline
		Algorithms& C &Sage\\
		\hline
		Setup   & 0.2856898 & 6.5858234\\
		Encrypt & 0.0061584 & 7.6450206\\
		Decrypt & 0.00951 & 3.3274426\\
		\hline
	\end{tabular}
\caption{Table de comparaison des temps d'exécution pour les différents algorithmes de Certificateless Cryptography }
\label{table:comparisonTimeAlgo}
\end{table}

\subsection{Librairie cryptographique}
%TODO + d'explications sur les librairies et les choix faits (retrouver l'endoit où l'on dit que RELIC est plus efficient et est plus fait pour les POCs etc)
La librairie utilisée est RELIC Toolkit~\cite{relic-toolkit}, c'est une librairie en cours de développement qui se veut efficiente. Sa concurrence avec MIRACL m'a fait hésiter dans mon choix, mais MIRACL est plus codée en~ C++ avec des équivalences en C j'ai donc choisi RELIC. De plus j'ai trouvé par exemple ici~\cite{bibid} que RELIC était généralement plus adapté dans le domaine universitaire pour des POC 
\subsection{Courbe utilisée}
La courbe utilisée pour le POC est la BLS12-P381, en effet cette courbe est assez efficiente et compatible avec les pairings. De plus RELIC l'a dans ses options et fonctionne bien, elle a un niveau de sécurité de 128bits. Je voulais prendre une courbe avec une plus grande sécurité cependant RELIC ne l'a pas encore totalement implémenté (certains tests concernant $\mathbb{G}_2$ ne passes pas), mais la librairie étant toujours en cours de développement il faudrait suivre ça de près, le code ne changerait en effet pas.
% TODO voir si KDF n'st pas mieux ? ou même générique de libsodium
\subsection{Dérivation de la clé AES}
Le but de mon schéma certificateless est de chiffrer puis signer une clé AES qui permettra à mon message d'avoir un chiffrement authentifié. Pour cela il me faut dériver un élément de $\mathbb{G}_t$ en clé AES, en effet le chiffrement dans le schéma certificateless se fait sur un élément de $\mathbb{G}_t$.\\
Pour cela j'ai utilisé une fonction permettant d'écrire sous forme compressée cet élément en bytes (fourni par la librairie RELIC utilisé et la fonction gt\_write\_bin()). Puis j'ai effectué un hachage avec SHA256 dessus, ainsi le résultat du hachage est une clé de 256 bits utilisable par AES-256-GCM. La fonction de hachage doit être par conséquent cryptographiquement sûre.
\subsection{Fonctions de hachage - signature}
Pour le schéma de signature il nous faut plusieurs fonctions de hachage différentes, en effet ce shéme est basé sur le \textit{Random Oracle Model} comme définit dans le chapitre \ref{ch:analysis}. Pour appliquer cela j'ai utilisé la même méthode de mapping disponible ans RELIC pour mapper une char array (tableau de byte) à un point sur G2 à savoir g2\_map.
Pour H1, la première fonction de hachage j'ai simplement utilisé cette fonction directement, mais pour H2 et H3 j'ai ajouté un byte devant les données à mapper respectivement les bytes '01' et '02'. Ceci afin de séparer les domaines des résultats des hashs, cela s'appelle du \textit{Hash Domain Separation}. En effet l'on peut voir dans ce draft~\cite{irtf-cfrg-hash-to-curve} définit comme une simulation pour prendre en compte plusieurs \textit{Random Oracle}.
\subsection{Sérialisation des données}
Pour la sérialisation des données, typiquement les clés publiques et les clés privées partielles envoyées en réseau ou les clés publiques enregistrées dans les fichiers par exemple, j'ai utilisé la librairie binn\footnote{\url{https://github.com/liteserver/binn}}. Cela permet de packer facilement des données binaires, pour cela RELIC met à disposition des méthodes g1\_write\_bin g1\_read\_bin qui a permis de faire ces enregistrements binaires. Ainsi les transferts de données sont simplifiés. Cependant il faut faire attention à certaines choses, on ne peut lire et écrire simultanément à l'aide de bin, si l'on crée un objet via un buffer on ne pourra modifier cet objet. Cela m'a posé des problèmes pour l'enregistrement des données secrétes, j'ai donc du copier l'objet lu pour pouvoir le modifier et sauver les nouveau paramétres.
\subsection{Enregistrement des clés publiques (serveur)}
Pour l'enregistrement j'ai utilisé une petite base de données NoSQL stockant les clés publiques des utilisateurs sur le KGC. Cela permet de facilement récupérer une clé publique pour un utilisateur si besoin. Pour implémenter cela j'ai utilisé la librairie UnQlite\footnote{\url{https://unqlite.org/}}. J'ai stocké les clés publiques pour le schéma de signature et de chiffrement séparemment, en effet, l'entrée pour la signature porte le nom "signature/ID" et le chiffrement "encryption/ID".
\section{Implémentation clés  de chiffrement}
Pour pouvoir implémenter ce schéma de chiffrement et signature certificateless dans un système hybride il a fallu penser à une manière d'encapsuler la clé et les données. Pour cela j'ai essayé de faire un système comparable à la figure \ref{fig:encapsulate}. 
\begin{figure}[h!]
	\centering
	\includegraphics[width=12cm]{images/schemaEncapsulation.png}
	\caption{Schéma encapsulation des données}
	\label{fig:encapsulate}
\end{figure}

\section{Fonctionnement global POC (KGC)}
Ici je présente le fonctionnement global de mon implémentation du KGC pour mon POC. De plus je présente les problèmes connus et des propositions d'améliorations.
\subsection{Fonctionnement}
\subsection{Problèmes connus}
\subsection{Améliorations}
\section{Fonctionnement global POC (Client)}
Ici je présente le fonctionnement global de l'implémentation du client mail pour mon POC, des améliorations possibles et des problèmes connus.
\subsection{Fonctionnement}
\paragraph*{Sécurité connexion mail.}
%TODO : présenter des captures de paquets chiffrées de connexion IMAP et SMTP
\subsection{Problèmes connus}
\subsection{Améliorations}
\paragraph*{Multiples destinataires.}
\section{Comparaisons avec état de l'art}
Dans cette section je vais présenter les différents protocoles et implémentations existantes présentées au chapitre \ref{ch:analysis} et les comparer à mon implémentation. Tout d'abord en présentant les différentes propriétés cryptographiques puis les temps d'exécution.

% TODO à analyser et remplir correctement
\subsection{Propriétés cryptographiques}
\begin{table}[h!]
	\centering
	\begin{tabular}{ |p{3cm}||p{3cm}|p{3cm}| }
		\hline
		\multicolumn{3}{|c|}{Comparaisons des propriétés cryptographiques proposées} \\
		\hline
		Implémentations & E2EE & Forward Secrecy\\
		\hline
		CLPKC-POC   & Oui & Oui\\
		PGP & Oui & Non\\
		S/MIME & Oui & Non\\
		\hline
	\end{tabular}
	\caption{Table de comparaison des différentes propriétés cryptographiques }
	\label{table:comparisonProperties}
\end{table}
% TODO si le temps le permet faire des calculs sur les temps de chiffrement / déchiffrement
\subsection{Temps des différentes implémentations}
Comparasion du temps mis pour chiffrer et signer / déchiffrer et vérifier un mail entres les différentes implémentations existantes.
\begin{table}[h!]
	\centering
	\begin{tabular}{ |p{3cm}||p{3cm}|p{3cm}| }
		\hline
		\multicolumn{3}{|c|}{Comparaisons des temps d'exécution entres différentes implémentations proposées} \\
		\hline
		Implémentations & Chiffrement & Déchiffrement\\
		\hline
		CLPKC-POC   & 0.0061584s & 0.00951s\\
		PGP & 0 & 0\\
		S/MIME & 0 & 0\\
		\hline
	\end{tabular}
	\caption{Table de comparaison des temps d'exécution entres les implémentations de mails chiffrés}
	\label{table:comparisonTime}
\end{table}
\subsection{Overhead induit}
Ici je présente les différents \textit{overhead} que j'ai remarqué en utilisant les différents systèmes de mails proposés.
\begin{table}[h!]
	\centering
	\begin{tabular}{ |p{3cm}||p{5cm}|p{6cm}| }
		\hline
		\multicolumn{3}{|c|}{Comparaisons de l'overhead induit dans un mail} \\
		\hline
		Implémentations & Taille overhead & Contenu\\
		\hline
		CLPKC-POC   & Environ 1200 bytes & Signature, timestamp, nonce, Encrypted Session Key\\
		PGP & Environ 300 bytes & Encrypted Session key\\
		S/MIME & 0 & jE SAIS PAS\\
		\hline
	\end{tabular}
	\caption{Table de comparaison des différents overhead en rapport avec les solutions existantes }
	\label{table:comparisonOverhead}
\end{table}

\chapter{Conclusion}
\label{ch:conclusion}
Lors de ce travail de bachelor la problématique principale 
\pagebreak

% +---------------------------------------------------------------+
\cleardoublepage
\addcontentsline{toc}{chapter}{Bibliographie}
\renewcommand{\thechapter}{}
\renewcommand{\chaptername}{}
\chaptermark{Bibliographie}
\bibliographystyle{plain}
\bibliography{chapters/biblio}
\nocite{*} %ajoute tout ce qu'il y a dans le bibtex
\cleardoublepage
\chaptermark{Liste des Figures}
\listoffigures
\cleardoublepage
\chaptermark{Liste des tableaux}
\listoftables

% Annexes
% +---------------------------------------------------------------+
\appendix

\chapter{Outils utilisés pour la compilation}

\section{RELIC Toolkit}
Pour pouvoir faire des calculs de \textit{Pairings} et sur des courbes elliptiques je me suis fier à RELIC Toolkit~\cite{relic-toolkit} qui est une librairie C permettant ce genre de calculs assez simplement. Le choix de la librairie est expliquée plus en détails dans le chapitre \ref{ch:impl}.\\
Cette librairie demande à être compilée avec une certaine courbe et certaines options (typiquement fonction de hachage et autres...). Des presets existent et c'est donc ce que j'ai utilisé pour ce POC. Cela demande donc de fournir la librairie précompilée avec les bonnes options pour l'utilisateur. L'inconvénient c'est donc que pour mettre à jour une courbe il va falloir recompiler toute la librairie et la fournir à l'utilisateur.
\section{Libsodium}
Pour faire du chiffrement authentifié l'utilisation de la librairie  libsodium\footnote{https://libsodium.gitbook.io/doc/} était le premier choix. En effet, la librairie ne m'est pas totalement étrangère et est très complète au niveau du chiffrement authentifié, gestion de la mémoire sécurisée, algorithmes de dérivation de clés et d'autres. Ainsi le POC nécessite d'avoir libsodium installé sur la machine exécutant le client du POC.
\section{binn}
Binn\footnote{\url{https://github.com/liteserver/binn}} est une librairie de sérialisation en C, permettant de créer des objets, des maps et des listes. Cela a été très utile pour la sérialisation des données entres le serveur et le client du POC et lors de l'enregistrement des différents paramétres du côté du client. Le POC utilise cette librairie et il faut donc avoir la librairie installée pour la compilation.
\section{libetpan}
LibetPan\footnote{\url{https://www.etpan.org/libetpan.html}} est une librairie visant à simplifier la gestion des emails dans le code C. Ainsi cette librairie est utilisé pour la récupération des emails depuis le serveur GMailmais aussi pour \textit{parser} les mails reçus et analyser le contenu afin de pouvoir les déchiffrer. Cette librairie aurrait pu être utilisée pour l'envoi des mails mais la connaissance de cette librairie est parvenue après libcurl.
\section{libcurl}
Libcurl\footnote{\url{https://curl.haxx.se/libcurl/}} permet d'envoyer toute sorte de requêtes simplement à l'aide d'une API de programmation assez simple et efficace. Dans le POC elle est utilisée afin d'envoyer des requêtes SMTPs pour envoyer des emails chiffrés générés par le POC.
\section{UnQlite}
UnQlite\footnote{\url{https://unqlite.org/}} permet d'avoir une base de données NoSQL gérée en C et ainsi ne pas avoir de base de données complexes pour une utilisation simple comme dans ce cas. En effet, lors de l'enregistrement des clés publiques des utilisateurs il suffit juste de pouvoir retrouver tel clé par rapport à tel ID. Ainsi une base de données avec ce genre d'entrées suffit. Sur leur site UnQlite propose une version précompilée de la librairie payante, cependant le github\footnote{\url{https://github.com/symisc/unqlite}} est publique et l'on peut donc le compiler de nous même. Ce qui doit d'ailleurs être fait dans le cadre de ce POC, en effet l'instance UnQlite fonctionnant sur le KGC et le KGC fonctionnant dans un environnement multi-threadé il est nécessaire de compiler la librairie avec un flag de compilation activant le support multiThread.

\chapter{Fichiers}
\label{ch:fichiers}
Liste des fichiers annexes au rapport, ce qu'ils contiennent et comment les utiliser si besoin.
\section{Code du \textit{Proof Of Concept}}
Le code est en annexe du rapport avec le nom \textit{POCCertificateless}.
Le \textit{README.md} présent à la source devrait être suffisant pour compiler soi-même le code. De plus, un repository github contenant le code, testé sur une installation Ubuntu 20.04 est disponible à l'adresse \url{https://github.com/mbonjour/POCCertificateless}.

Le code est commenté à la manière \textit{Doxygen} afin d'avoir des informations à chaque fois qu'une fonction est référencée. De plus, le code est commenté afin de comprendre le déroulement global des différentes parties. Le code va générer ainsi trois exécutables dont le client, le serveur et un test afin de calculer le temps mis par les algorithmes et utilisé anciennement pour se représenter le déroulement global d'un échange entre Alice et Bob.
\section{Tableaux comparatifs}
Les tableaux comparatifs cités dans le chapitre \ref{ch:analysis} apparaissent sous forme de feuille dans un fichier excel se nommant \textit{ComparatifCLESchemes.xlsx}. 
La feuille nommée CLEs contient un comparatif des schémas de chiffrement tandis que la feuille CLS contient un comparatif des schémas de signatures.

Ces tableaux sont mis en annexes car il représente une grande quantité de données qui apparaissait mal dans ce rapport.


\end{document}
